% Options for packages loaded elsewhere
\PassOptionsToPackage{unicode}{hyperref}
\PassOptionsToPackage{hyphens}{url}
%
\documentclass[
]{book}
\usepackage{lmodern}
\usepackage{amssymb,amsmath}
\usepackage{ifxetex,ifluatex}
\ifnum 0\ifxetex 1\fi\ifluatex 1\fi=0 % if pdftex
  \usepackage[T1]{fontenc}
  \usepackage[utf8]{inputenc}
  \usepackage{textcomp} % provide euro and other symbols
\else % if luatex or xetex
  \usepackage{unicode-math}
  \defaultfontfeatures{Scale=MatchLowercase}
  \defaultfontfeatures[\rmfamily]{Ligatures=TeX,Scale=1}
\fi
% Use upquote if available, for straight quotes in verbatim environments
\IfFileExists{upquote.sty}{\usepackage{upquote}}{}
\IfFileExists{microtype.sty}{% use microtype if available
  \usepackage[]{microtype}
  \UseMicrotypeSet[protrusion]{basicmath} % disable protrusion for tt fonts
}{}
\makeatletter
\@ifundefined{KOMAClassName}{% if non-KOMA class
  \IfFileExists{parskip.sty}{%
    \usepackage{parskip}
  }{% else
    \setlength{\parindent}{0pt}
    \setlength{\parskip}{6pt plus 2pt minus 1pt}}
}{% if KOMA class
  \KOMAoptions{parskip=half}}
\makeatother
\usepackage{xcolor}
\IfFileExists{xurl.sty}{\usepackage{xurl}}{} % add URL line breaks if available
\IfFileExists{bookmark.sty}{\usepackage{bookmark}}{\usepackage{hyperref}}
\hypersetup{
  pdftitle={কুরআন স্টাডি},
  pdfauthor={আহমদ বিন রফিক},
  hidelinks,
  pdfcreator={LaTeX via pandoc}}
\urlstyle{same} % disable monospaced font for URLs
\usepackage{longtable,booktabs}
% Correct order of tables after \paragraph or \subparagraph
\usepackage{etoolbox}
\makeatletter
\patchcmd\longtable{\par}{\if@noskipsec\mbox{}\fi\par}{}{}
\makeatother
% Allow footnotes in longtable head/foot
\IfFileExists{footnotehyper.sty}{\usepackage{footnotehyper}}{\usepackage{footnote}}
\makesavenoteenv{longtable}
\usepackage{graphicx,grffile}
\makeatletter
\def\maxwidth{\ifdim\Gin@nat@width>\linewidth\linewidth\else\Gin@nat@width\fi}
\def\maxheight{\ifdim\Gin@nat@height>\textheight\textheight\else\Gin@nat@height\fi}
\makeatother
% Scale images if necessary, so that they will not overflow the page
% margins by default, and it is still possible to overwrite the defaults
% using explicit options in \includegraphics[width, height, ...]{}
\setkeys{Gin}{width=\maxwidth,height=\maxheight,keepaspectratio}
% Set default figure placement to htbp
\makeatletter
\def\fps@figure{htbp}
\makeatother
\setlength{\emergencystretch}{3em} % prevent overfull lines
\providecommand{\tightlist}{%
  \setlength{\itemsep}{0pt}\setlength{\parskip}{0pt}}
\setcounter{secnumdepth}{5}
\usepackage[]{natbib}
\bibliographystyle{apalike}

\title{কুরআন স্টাডি}
\author{আহমদ বিন রফিক}
\date{2021-07-19}

\begin{document}
\maketitle

{
\setcounter{tocdepth}{1}
\tableofcontents
}
\hypertarget{about-quran}{%
\chapter*{কুরআনের পরিচয়}\label{about-quran}}
\addcontentsline{toc}{chapter}{কুরআনের পরিচয়}

\hypertarget{why-quran-was-revealed}{%
\section*{কুরআন নাযিলের উদ্দেশ্য}\label{why-quran-was-revealed}}
\addcontentsline{toc}{section}{কুরআন নাযিলের উদ্দেশ্য}

কুরআন এসেছে মানবজাতিকে অন্ধকার থেকে মুক্ত করে আলোর পথ দেখানোর জন্যে। সুরা ইব্রাহীমে আল্লাহ বলেন:

\begin{quote}
الر ۚ كِتَابٌ أَنزَلْنَاهُ إِلَيْكَ لِتُخْرِجَ النَّاسَ مِنَ الظُّلُمَاتِ إِلَى النُّورِ بِإِذْنِ رَبِّهِمْ إِلَىٰ صِرَاطِ الْعَزِيزِ الْحَمِيدِ
\end{quote}

\begin{quote}
আলিফ লাম রা। আমি এ কিতাব নাযিল করেছি যাতে আপনি মানুষকে তাদের রবের নির্দেশে অন্ধকার থেকে বের করে আলোর দিকে মহাপরাক্রমশালী ও মহাপ্রশংসিত রবের পথের দিকে নিয়ে যেতে পারেন।

--- সুরা ইব্রাহীম ১৪:১
\end{quote}

\hypertarget{surah}{%
\chapter*{সুরাহসমূহ}\label{surah}}
\addcontentsline{toc}{chapter}{সুরাহসমূহ}

এ অংশে থাকবে সুরাহগুলোর নামের অর্থ, নাযিলের সময়কাল ও সংক্ষিপ্ত আলোচ্যসূচি।

\hypertarget{quran-index}{%
\chapter*{বিষয়ভিত্তিক কুরআন}\label{quran-index}}
\addcontentsline{toc}{chapter}{বিষয়ভিত্তিক কুরআন}

\hypertarget{traveling}{%
\section*{পৃথিবী ভ্রমণের নির্দেশ}\label{traveling}}
\addcontentsline{toc}{section}{পৃথিবী ভ্রমণের নির্দেশ}

সুরাহ আলে ইমরান:

\begin{quote}
قَدۡ خَلَتۡ مِنۡ قَبۡلِكُمۡ سُنَنٌۙ فَسِيۡرُوۡا فِىۡ الۡاَرۡضِ فَانظُرُوۡا كَيۡفَ كَانَ عَاقِبَةُ الۡمُكَذِّبِيۡنَ
\end{quote}

\begin{quote}
অর্থ: তোমাদের আগে অনেক জীবনপদ্ধতি অতিবাহিত হয়েছে। পৃথিবীতে ভ্রমণ করে দেখোই না মিথ্যা প্রতিপন্নকারীদের পরিণাম কেমন হয়েছে।

--- \href{https://tanzil.net/\#3:137}{সুরাহ আলে ইমরান, আয়াত ৩:১৩৭}
\end{quote}

সুরাহ আল মায়িদা:

\begin{quote}
قُلْ سِيرُوا فِي الْأَرْضِ ثُمَّ انظُرُوا كَيْفَ كَانَ عَاقِبَةُ الْمُكَذِّبِينَ
\end{quote}

\begin{quote}
অর্থ: বলুন, পৃথিবীতে ঘুরে দেখে নাও মিথ্যা প্রতিপন্নকারীদের পরিণাম কেমন ছিল।

--- \href{https://tanzil.net/\#6:11}{সুরাহ আল আন'আম, আয়াত ৬:১০-১১}
\end{quote}

আরও দেখুন \href{https://alomoy.github.io/islam/travelling.al}{এখানে}

\end{document}
