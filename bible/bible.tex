% Options for packages loaded elsewhere
\PassOptionsToPackage{unicode}{hyperref}
\PassOptionsToPackage{hyphens}{url}
%
\documentclass[
]{book}
\usepackage{lmodern}
\usepackage{amssymb,amsmath}
\usepackage{ifxetex,ifluatex}
\ifnum 0\ifxetex 1\fi\ifluatex 1\fi=0 % if pdftex
  \usepackage[T1]{fontenc}
  \usepackage[utf8]{inputenc}
  \usepackage{textcomp} % provide euro and other symbols
\else % if luatex or xetex
  \usepackage{unicode-math}
  \defaultfontfeatures{Scale=MatchLowercase}
  \defaultfontfeatures[\rmfamily]{Ligatures=TeX,Scale=1}
\fi
% Use upquote if available, for straight quotes in verbatim environments
\IfFileExists{upquote.sty}{\usepackage{upquote}}{}
\IfFileExists{microtype.sty}{% use microtype if available
  \usepackage[]{microtype}
  \UseMicrotypeSet[protrusion]{basicmath} % disable protrusion for tt fonts
}{}
\makeatletter
\@ifundefined{KOMAClassName}{% if non-KOMA class
  \IfFileExists{parskip.sty}{%
    \usepackage{parskip}
  }{% else
    \setlength{\parindent}{0pt}
    \setlength{\parskip}{6pt plus 2pt minus 1pt}}
}{% if KOMA class
  \KOMAoptions{parskip=half}}
\makeatother
\usepackage{xcolor}
\IfFileExists{xurl.sty}{\usepackage{xurl}}{} % add URL line breaks if available
\IfFileExists{bookmark.sty}{\usepackage{bookmark}}{\usepackage{hyperref}}
\hypersetup{
  pdftitle={বাইবেল স্টাডি},
  pdfauthor={আহমদ বিন রফিক},
  hidelinks,
  pdfcreator={LaTeX via pandoc}}
\urlstyle{same} % disable monospaced font for URLs
\usepackage{longtable,booktabs}
% Correct order of tables after \paragraph or \subparagraph
\usepackage{etoolbox}
\makeatletter
\patchcmd\longtable{\par}{\if@noskipsec\mbox{}\fi\par}{}{}
\makeatother
% Allow footnotes in longtable head/foot
\IfFileExists{footnotehyper.sty}{\usepackage{footnotehyper}}{\usepackage{footnote}}
\makesavenoteenv{longtable}
\usepackage{graphicx,grffile}
\makeatletter
\def\maxwidth{\ifdim\Gin@nat@width>\linewidth\linewidth\else\Gin@nat@width\fi}
\def\maxheight{\ifdim\Gin@nat@height>\textheight\textheight\else\Gin@nat@height\fi}
\makeatother
% Scale images if necessary, so that they will not overflow the page
% margins by default, and it is still possible to overwrite the defaults
% using explicit options in \includegraphics[width, height, ...]{}
\setkeys{Gin}{width=\maxwidth,height=\maxheight,keepaspectratio}
% Set default figure placement to htbp
\makeatletter
\def\fps@figure{htbp}
\makeatother
\setlength{\emergencystretch}{3em} % prevent overfull lines
\providecommand{\tightlist}{%
  \setlength{\itemsep}{0pt}\setlength{\parskip}{0pt}}
\setcounter{secnumdepth}{5}
\usepackage[]{natbib}
\bibliographystyle{apalike}

\title{বাইবেল স্টাডি}
\author{আহমদ বিন রফিক}
\date{2021-07-24}

\begin{document}
\maketitle

{
\setcounter{tocdepth}{1}
\tableofcontents
}
\hypertarget{about-bible}{%
\chapter*{বাইবেল কী?}\label{about-bible}}
\addcontentsline{toc}{chapter}{বাইবেল কী?}

শাব্দিকভাবে বাইবেল অর্থ বইসমূহ। আসলেই বাইবেল অনেকগুলো বইয়ের সমাবেশ। বর্তমান খৃষ্টীয় বাইবেল দুটি ভাগে বিভক্ত। ওল্ড ও নিউ টেস্টামেন্ট (Old \& New Testament)। প্রোটেস্টেন্ট খ্রিষ্টানদের মতে ওল্ড টেস্টামেন্টে ৩৯টি বই আছে। আর ক্যাথলিকরা বলেন, বই আছে ৪৬টি। ওল্ড টেস্টামেন্টের প্রথম ৫টি বইয়ের নাম যথাক্রমে জেনেসিস, এক্সোডাস, লেভিটিকাস, নাম্বারস ও ডিউটোরনমি। এই পাঁচটি আবার একত্রে বলা হয় তাওরাত। যেটা নাযিল হয়েছিল হযরত মুসার (আ) ওপর। বলাই বাহুল্য, এই প্রথম পাঁচটি বইই ইহুদিদের মূল ধর্মগ্রন্থ। মুসার (আ) পরের কোনো নবীকে তারা স্বীকার করে না। আবার হযরত ঈসা (আ) নতুন কোনো আইন নিয়ে আসেননি (আল্লাহর ইচ্ছাতেই) বলে হযরত মুসা (আ) ও তাঁর পরে ও ঈসার (আ) আগে আসা সকল নবীদের আসমানী কিতবাগুলো হযরত ঈসার (আ) উম্মতের জন্য প্রযোজ্য ছিল।

আর এদিকে নিউ টেস্টামেন্ট হলো আমরা মুসলমানরা যেটাকে ইঞ্জিল বলে জানি। এতে আছে ২৭টি বই। প্রধান চারটি বই (গসপেল) লিখেছেন মথি (Mathew),মার্ক (Mark), লূক (Luke)ও যোহান (John)।

\hypertarget{islam-on-christians}{%
\subsection*{খ্রিষ্টানদেরকে ইসলাম কীভাবে দেখে?}\label{islam-on-christians}}
\addcontentsline{toc}{subsection}{খ্রিষ্টানদেরকে ইসলাম কীভাবে দেখে?}

আল্লাহ সুরাহ মায়িদায় বলেন:

\begin{quote}
وَلَتَجِدَنَّ اَقۡرَبَهُمۡ مَّوَدَّةً لِّلَّذِيۡنَ اٰمَنُوۡا الَّذِيۡنَ قَالُوۡۤا اِنَّا نَصٰرٰى‌ؕ ذٰلِكَ بِاَنَّ مِنۡهُمۡ قِسِّيۡسِيۡنَ وَرُهۡبَانًا وَّاَنَّهُمۡ لَا يَسۡتَكۡبِرُوۡنَ
\end{quote}

\begin{quote}
অর্থ: যারা ঈমান এনেছে আপনি তাদের সাথে বন্ধুত্বের ব্যাপারে সবচেয়ে নিকটে পাবেন তাদেরকে যারা বলে, 'আমরা খ্রিষ্টান বা নাসারা (সাহায্যকারী)। এর কারণ তাদের মধ্যে রয়েছে আলেম ও সংসারবিরাগী দরবেশ। আর তারা অহঙ্কার করে না।

--- \href{http://tafheembangla.com/index.php/quran?show=quran\&surah_no=5\&limitstart=81}{সুরাহ আল মায়িদা ৫:৮২}
\end{quote}

এভাবে ইসলাম বলছে, মুসলমানদের জন্যে বন্ধুত্বের দিক থেকে খ্রিষ্টানরাই সবচেয়ে নিকটে। এ কারণেই রাসুল (সা) মক্কী জীবনের সময়ে সংঘটিত রোম-পারস্যের যুদ্ধে অগ্নিপূজক ইরানিদের বদলে খ্রিষ্টধর্মের রোমকদের বিজয় কামনা করেছিলেন।

এছাড়াও আহলে কিতাবদের মধ্যেও অনেকেই মুহাম্মাদের (সা) বাণীর সত্যতা বুঝতে পেরে তাঁর ওপর ঈমান আনে।

\begin{quote}
وَاِنَّ مِنۡ اَهۡلِ الۡكِتٰبِ لَمَنۡ يُّؤۡمِنُ بِاللّٰهِ وَمَاۤ اُنۡزِلَ اِلَيۡكُمۡ وَمَاۤ اُنۡزِلَ اِلَيۡهِمۡ خٰشِعِيۡنَ لِلّٰهِۙ لَا يَشۡتَرُوۡنَ بِاٰيٰتِ اللّٰهِ ثَمَنًا قَلِيۡلاً‌ؕ اُولٰٓٮِٕكَ لَهُمۡ اَجۡرُهُمۡ عِنۡدَ رَبِّهِمۡ‌ؕ اِنَّ اللّٰهَ سَرِيۡعُ الۡحِسَابِ
\end{quote}

\begin{quote}
অর্থ: আহলে কিতাবদের মধ্যেও এমন লোক আছে যারা আল্লাহর উপর ঈমান আনে। এছাড়াও ঈমান আনে যা তোমার প্রতি নাযিল করা হয়েছে তার প্রতি এবং যা তাদের প্রতি নাযিল করা হয়েছে তার প্রতি। তারা আল্লাহর সামনে বিনয়াবনত থাকে। তারা আল্লাহর আয়াতকে সামান্য দামে বিক্রি করে না। তাদের জন্যে তাদের প্রতিপালকের কাছে রয়েছে প্রতিদান। আল্লাহ হিসাব গ্রহণে দেরি করেন না।

--- \href{http://tafheembangla.com/index.php/quran?show=quran\&surah_no=3\&limitstart=198}{সুরাহ আলে ইমরান ৩:১৯৯}
\end{quote}

\hypertarget{following-bible}{%
\section*{মুসলমানদেরকে কি বাইবেল মানতে হবে?}\label{following-bible}}
\addcontentsline{toc}{section}{মুসলমানদেরকে কি বাইবেল মানতে হবে?}

পবিত্র কুরআন নাযিলের সাথে সাথে আগের সকল আসমানী কিতাব রহিত হয়ে গেছে। কুরআনের সাথে কিছু মিলে গেলে সেটা কুরআনে আছে বলে তবেই মানতে হবে। {[}সুরা সফের ৬ নং আয়াতে{]} \url{https://tanzil.net/\#61:6}) আল্লাহ বলেছেন, ঈসাকে (আ) পাঠানো হয়েছিল বনী ঈসরাইলদের কাছে। কাজেই ঈসার (আ) কথা ও কাজ আমাদের জন্য অনুসরণীয় নয়, যদি না তাঁর কোনো নির্দেশনা মুহাম্মদের (সা) কথা ও কাজের সাথে মিলে যায়।

\hypertarget{bible-index}{%
\chapter*{বিষয়ভিত্তিক বাইবেল}\label{bible-index}}
\addcontentsline{toc}{chapter}{বিষয়ভিত্তিক বাইবেল}

বাইবেলের মূল সংস্করণ তো আল্লাহরই বাণী ছিল। বর্তমান বাইবেলকে আমরা মুসলমানরা আল্লাহর বাণী বলে স্বীকার করি না। অবশ্য বহু বিকৃতির পরেও বাইবেলে আল্লাহর কিছু বাণী অবশিষ্ট আছে বলেই মনে হয়। এছাড়াও সন নবীরাই আল্লাহর একত্ববাদসহ কিছু মৌলিক বিষয় একই আকারেই প্রচার করেছেন। এখানে বাইবেল থেকে প্রাপ্ত মূলত সে অংশগুলোই সংকলন করা হয়েছে। এছাড়াও এমন কিছু কথাও সংকলন করা হয়েছে যেগুলো নিয়ে চিন্তাভাবনা করলে আমাদের খ্রিষ্টান ভাইদের চোখ খুলে যাবে। ওরাকা বিন নওফলের মতো করে তাঁরা সত্যকে বুঝতে পারবেন।

\hypertarget{similarities-between-islam-and-christianity}{%
\section*{ইসলাম ও খৃষ্ট ধর্মের সাদৃশ্য}\label{similarities-between-islam-and-christianity}}
\addcontentsline{toc}{section}{ইসলাম ও খৃষ্ট ধর্মের সাদৃশ্য}

\hypertarget{jesus-muslim}{%
\section*{ঈসা (আ) মুসলমান ছিলেন}\label{jesus-muslim}}
\addcontentsline{toc}{section}{ঈসা (আ) মুসলমান ছিলেন}

গসপেল অব জন ৫ম অধ্যায়ের ৩০ অনুচ্ছেদে যিশু (আ) বলেন:

\begin{quote}
আমি নিজে থেকে কিছুই করতে পারি না; যেমন শুনি তেমন বিচার করি। আমি ন্যায়ভাবে বিচার করি, কারণ আমি আমার ইচ্ছামতো কাজ করতে চাই না, যিনি আমাকে পাঠিয়েছেন তাঁর ইচ্ছামতো কাজ করতে চাই।

--- ইঞ্জিল শরীফ, বাংলাদেশ বাইবেল সোসাইটি (দ্বিতীয় সংস্করণ)
\end{quote}

ইংরেজি সংস্করণ

\begin{quote}
By myself I can do nothing; I judge only as I hear, and my judgment is just, for I seek not to please myself but him who sent me.

--- \href{https://www.biblegateway.com/passage/?search=John+5\&version=NIV}{Bible Gateway (NIV)}
\end{quote}

যে বলে, `আমি নিজের ইচ্ছায় নয়, আল্লাহর ইচ্ছামতো কাজ করি তাকে বলা হয় মুসলিম।' ঈসা (আ) ছিলেন মুসলিম।

\end{document}
